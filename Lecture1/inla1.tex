% Options for packages loaded elsewhere
\PassOptionsToPackage{unicode}{hyperref}
\PassOptionsToPackage{hyphens}{url}
%
\documentclass[
  ignorenonframetext,
]{beamer}
\title{Bayesian Statistics with R-INLA}
\subtitle{University of Zurich, March, 2022}
\author{Instructor: Sara Martino}
\date{}

\usepackage{pgfpages}
\setbeamertemplate{caption}[numbered]
\setbeamertemplate{caption label separator}{: }
\setbeamercolor{caption name}{fg=normal text.fg}
\beamertemplatenavigationsymbolsempty
% Prevent slide breaks in the middle of a paragraph
\widowpenalties 1 10000
\raggedbottom
\setbeamertemplate{part page}{
  \centering
  \begin{beamercolorbox}[sep=16pt,center]{part title}
    \usebeamerfont{part title}\insertpart\par
  \end{beamercolorbox}
}
\setbeamertemplate{section page}{
  \centering
  \begin{beamercolorbox}[sep=12pt,center]{part title}
    \usebeamerfont{section title}\insertsection\par
  \end{beamercolorbox}
}
\setbeamertemplate{subsection page}{
  \centering
  \begin{beamercolorbox}[sep=8pt,center]{part title}
    \usebeamerfont{subsection title}\insertsubsection\par
  \end{beamercolorbox}
}
\AtBeginPart{
  \frame{\partpage}
}
\AtBeginSection{
  \ifbibliography
  \else
    \frame{\sectionpage}
  \fi
}
\AtBeginSubsection{
  \frame{\subsectionpage}
}
\usepackage{amsmath,amssymb}
\usepackage{lmodern}
\usepackage{iftex}
\ifPDFTeX
  \usepackage[T1]{fontenc}
  \usepackage[utf8]{inputenc}
  \usepackage{textcomp} % provide euro and other symbols
\else % if luatex or xetex
  \usepackage{unicode-math}
  \defaultfontfeatures{Scale=MatchLowercase}
  \defaultfontfeatures[\rmfamily]{Ligatures=TeX,Scale=1}
\fi
% Use upquote if available, for straight quotes in verbatim environments
\IfFileExists{upquote.sty}{\usepackage{upquote}}{}
\IfFileExists{microtype.sty}{% use microtype if available
  \usepackage[]{microtype}
  \UseMicrotypeSet[protrusion]{basicmath} % disable protrusion for tt fonts
}{}
\makeatletter
\@ifundefined{KOMAClassName}{% if non-KOMA class
  \IfFileExists{parskip.sty}{%
    \usepackage{parskip}
  }{% else
    \setlength{\parindent}{0pt}
    \setlength{\parskip}{6pt plus 2pt minus 1pt}}
}{% if KOMA class
  \KOMAoptions{parskip=half}}
\makeatother
\usepackage{xcolor}
\IfFileExists{xurl.sty}{\usepackage{xurl}}{} % add URL line breaks if available
\IfFileExists{bookmark.sty}{\usepackage{bookmark}}{\usepackage{hyperref}}
\hypersetup{
  pdftitle={Bayesian Statistics with R-INLA},
  pdfauthor={Instructor: Sara Martino},
  hidelinks,
  pdfcreator={LaTeX via pandoc}}
\urlstyle{same} % disable monospaced font for URLs
\newif\ifbibliography
\setlength{\emergencystretch}{3em} % prevent overfull lines
\providecommand{\tightlist}{%
  \setlength{\itemsep}{0pt}\setlength{\parskip}{0pt}}
\setcounter{secnumdepth}{-\maxdimen} % remove section numbering
\ifLuaTeX
  \usepackage{selnolig}  % disable illegal ligatures
\fi

\begin{document}
\frame{\titlepage}

\begin{frame}[allowframebreaks]
  \tableofcontents[hideallsubsections]
\end{frame}
\begin{frame}
\end{frame}

\hypertarget{plan-of-the-course}{%
\section{Plan of the course}\label{plan-of-the-course}}

\begin{frame}{Plan for this 2-day course}
\protect\hypertarget{plan-for-this-2-day-course}{}
\begin{block}{Today}
\protect\hypertarget{today}{}
\begin{itemize}
\tightlist
\item
  \textbf{9:00-10:45}Introduction and basics concepts of INLA
\item
  \textbf{11:00-12:30} Practical session I
\end{itemize}

\textbf{\textcolor{red}{Lunch}}

\begin{itemize}
\tightlist
\item
  \textbf{13:30-15:00} R-INLA: Basics
\item
  \textbf{15:15-17:00} Practical session II
\end{itemize}
\end{block}
\end{frame}

\begin{frame}{Plan for this 2-day course}
\protect\hypertarget{plan-for-this-2-day-course-1}{}
\begin{block}{Tomorrow}
\protect\hypertarget{tomorrow}{}
\begin{itemize}
\tightlist
\item
  \textbf{9:00-10:45}??
\item
  \textbf{11:00-12:30} Practical session III
\end{itemize}

\textbf{\textcolor{red}{Lunch}}

\begin{itemize}
\tightlist
\item
  \textbf{13:30-15:00} ??
\item
  \textbf{15:15-16:00} Practical session IV
\end{itemize}
\end{block}
\end{frame}

\begin{frame}
\end{frame}

\hypertarget{introduction}{%
\section{Introduction}\label{introduction}}

\begin{frame}[fragile]{What is inla?}
\protect\hypertarget{what-is-inla}{}
\textbf{The short answer:}\\
\strut \\

\begin{quote}
INLA is a fast method to do Bayesian inference with latent Gaussian
models and \texttt{INLA} is an \texttt{R}-package that implements this
method with a flexible and simple interface.
\end{quote}

\pause

\textbf{The (much) longer answer:}

\begin{itemize}
\tightlist
\item
  Rue, Martino, and Chopin (2009) ``Approximate Bayesian inference for
  latent Gaussian models by using integrated nested Laplace
  approximations.'' \emph{JRSSB}
\item
  Rue, Riebler, Sørbye, Illian, Simpson, Lindgren (2017) ``Bayesian
  Computing with INLA: A Review.'' \emph{Annual Review of Statistics and
  Its Application}
\item
  Martino, Riebler ``Integrated Nested Laplace Approximations (INLA)''
  (2021) \emph{arXiv:1907.01248}
\end{itemize}
\end{frame}

\begin{frame}
\end{frame}

\begin{frame}
\end{frame}

\end{document}
